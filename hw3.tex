\documentclass[12pt,letterpaper,boxed]{hmcpset}

\usepackage[margin=1in]{geometry}

\usepackage{graphicx}
\usepackage{amsmath}

\name{}
\class{Math 70}
\assignment{Homework 3}
\duedate{02/06/2018}

\begin{document}

\problemlist{(7A) 6, 11. (7B) 3, 12, 14.}

\begin{problem}[6]
Make $\mathcal{P}_2(\textbf{R})$ into an inner product space by defining $$ \langle p,q \rangle = \int_{0}^{1} p(x)q(x) dx $$. Define $ T \in \mathcal{L}(\mathcal{P}_2(\textbf{R})) $ by $ T(a_0 + a_1 x + a_2 x^2) = a_1 x $. 
\begin{itemize}
  \item[(\textit{a})] Show that $T$ is not self-adjoint. 
  
  \item[(\textit{b})] The matrix of $T$ with respect to the basis $(1,x,x^2)$ is 
$$
\begin{pmatrix}
0&0&0\\
0&1&0\\
0&0&0\\
\end{pmatrix}.
$$ This matrix equals its conjugate transpose, even though $T$ is not self-adjoint. Explain why this is not a contradiction. 
\end{itemize}
\end{problem}

\begin{solution}

\end{solution}

\clearpage

\begin{problem}[11]
Suppose $ \mathcal{P} \in \mathcal{L}(V) $ is such that $ P^2 = P $. Prove that there is a subspace $U$ of $V$ such that $P = P_U$ if and only if $P$ is self-adjoint. 
\end{problem}

\begin{solution}
\end{solution}

\clearpage

\begin{problem}[3]
Give an example of an operator $ T \in \mathcal{L}(\textbf{C}^3) $ such that $2$ and $3$ are the only eigenvalues of $T$ and $ T^2 - 5T + 6I \neq 0 $.
\end{problem}

\begin{solution}
\end{solution}

\clearpage


\begin{problem}[12]
Suppose $ T \in \mathcal{L}(V) $ is self-adjoint, $\lambda \in \textbf{F}$, and $\epsilon > 0 $. Suppose there exists $v \in V$ such that $ \vert \vert v \vert \vert = 1 $ and $$ \vert \vert Tv - \lambda v \vert \vert < \epsilon. $$ Prove that $T$ has an eigenvalue $ \lambda'$ such that $ \vert \lambda - \lambda' \vert < \epsilon $.
\end{problem}

\begin{solution}
\end{solution}

\clearpage

\begin{problem}[14]
Suppose $U$ is a finite-dimensional real vector space and $ T \in \mathcal{L}(U) $. Prove that $U$ has a basis consisting of eigenvectors of $T$ if and only if there is an inner product on $U$ that makes $T$ into a self-adjoint operator. 
\end{problem}

\begin{solution}

\end{solution}

\end{document}
